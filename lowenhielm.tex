\documentclass[twoside, openright]{report}
\usepackage[utf8]{inputenc}
\usepackage[swedish]{babel}
\usepackage{geometry}
\usepackage{makeidx}
\usepackage{idxlayout}
\usepackage{sectsty}
\usepackage{gensymb}

\setlength{\parindent}{0pt}
\chapterfont{\Large}
\chaptertitlefont{\Large}
\sectionfont{\large}
\geometry{a6paper, textwidth=80mm, textheight=120mm}

\title{Greve Löwenhielms Sångbok Remastered}

\author{Special LinDa \texttt{0010 0000} edition}

\makeindex
\begin{document}
\maketitle
\thispagestyle{empty}

\cleardoublepage
\begin{abstract}
Det här är ett försök att återskapa den legendariska Greve Löwenhielms Sångbok i \LaTeX. Fö\-hoppningsvis så växer den alleftersom. Det är inte meningen att det ska vara en exakt kopia, tillägg är tillåtet.\\

Det finns ingen copyrightsida i den gamla sångboken som Karlstads studentkår gav ut så vi antar att det är fritt fram för vem som helst att fortsätta traditionen.\\

Den är typsatt för A6-papper men kan även skrivas ut som häfte på A5-papper. Kom med pdf-filen till Tryckeriet så fixar dom det säkert åt dig.\\

Är det någon sång som du vill läggs till så är det bara att göra det på:\\
\texttt{\small https://github.com/Lord-Wolfenstein/Lowenhielm/}
\end{abstract}
\thispagestyle{empty}
\tableofcontents

%%%%%%%%%%%%%%%%%%%%%%%%%%%%%%%%%%%%%%%%%%%%%%%%%%%%%%%%%%%%%%%%%%%%%%%%%%%%%%%%
%%%%%%%%%%%%%%%%%%%%%%%%%%%%%%%%%%%%%%%%%%%%%%%%%%%%%%%%%%%%%%%%%%%%%%%%%%%%%%%%
%%%%%%%%%%%%%%%%%%%%%%%%%%%%%%%%%%%%%%%%%%%%%%%%%%%%%%%%%%%%%%%%%%%%%%%%%%%%%%%%
\chapter*{Sittningsdiciplin}

\begin{enumerate}

\item När toastmastern talar är man TYST. När toastmastern påkallar uppmärksamhet har man tre ord på sig att avsluta pågående samtal. Dock om det är så att man har halva inne, må man nyttja fem ord för att avsluta konversationen.

\item När man tycker att toastmastern bör ta upp en sång påkallar man dennes
uppmärksamhet med: TEMPO!

\item När toastmastern tar upp en sång sjunger man; samma sång som toastmastern!

\item Brott mot tyngdlagen undanbedes. (Men om du lyckas så kontakta en proffesor på måndag och visa hur du gjorde så garanterar vi att du åtminståne får ett pris för det.)

\item Det är strängeligen förbjudet att behålla sitt goda humör för sig själv!

\item Intagen föda återtages ej.

\item När maten är slut på din egen tallrik, fortsätt på den till höger.

\item De som känner sig osäkra ifråga om sången borde hålla sig en takt före, så att när slutet av sången nås, ingen blir efter.

\item De som ej har medfödd sångröst äga ändå rätt att efter bästa förmåga delta i sången. Alla är dock skyldiga att sjunga hellre än bra.

\item Om man vill sjunga en sång på egen hand må detta endast i undantagsfall göras och i så fall ska det undantagslöst göras under bordet.

\item Dörrar med höga trösklar kallas fönster, dem får man inte gå ut genom.

\item Det är ej rekommenderat att luta sig över bordet när rummet flyttar sig i en cirkel.

\item Kontrollera att era bordsgrannar ej utnyttjar rusdrycker så de får suddiga ansikten.

\item Under bordet: Håll till vänster!

\end{enumerate}

%%%%%%%%%%%%%%%%%%%%%%%%%%%%%%%%%%%%%%%%%%%%%%%%%%%%%%%%%%%%%%%%%%%%%%%%%%%%%%%%
%%%%%%%%%%%%%%%%%%%%%%%%%%%%%%%%%%%%%%%%%%%%%%%%%%%%%%%%%%%%%%%%%%%%%%%%%%%%%%%%
%%%%%%%%%%%%%%%%%%%%%%%%%%%%%%%%%%%%%%%%%%%%%%%%%%%%%%%%%%%%%%%%%%%%%%%%%%%%%%%%
\chapter{Högtidliga sånger}

\section{Du gamla du fria}\index{Du gamla du fria}
Du gamla, du fria, du fjellhöga Nord,\\
Du tysta, du glädjerika sköna!\\
Jag helsar dig, vänsta land uppå jord,\\
Din sol, din himmel, dina ängder gröna.\\
Din sol, din himmel, dina ängder gröna.\\

Du tronar på minnen från fornstora da'r,\\
Då äradt ditt namn flög öfver jorden;\\
Jag vet att du är och blir hvad du var,\\
Ack, jag vill lefva, jag vill dö i Norden!\\
Ack, jag vill lefva, jag vill dö i Norden!

\section{Ack, Värmland du sköna}\index{Ack, Värmland du sköna}
Ack, Värmland, du sköna, du härliga land,\\
du krona bland Svea rikes länder!\\
Och komme jag än mitt i det förlovade land,\\
till Värmland jag ändå återvänder.\\
Ja, där vill jag leva, ja, där vill jag dö.\\
Om en gång ifrån Värmland jag tager mig en mö,\\
så vet jag att aldrig jag mig ångrar.

%%%%%%%%%%%%%%%%%%%%%%%%%%%%%%%%%%%%%%%%%%%%%%%%%%%%%%%%%%%%%%%%%%%%%%%%%%%%%%%%
%%%%%%%%%%%%%%%%%%%%%%%%%%%%%%%%%%%%%%%%%%%%%%%%%%%%%%%%%%%%%%%%%%%%%%%%%%%%%%%%
%%%%%%%%%%%%%%%%%%%%%%%%%%%%%%%%%%%%%%%%%%%%%%%%%%%%%%%%%%%%%%%%%%%%%%%%%%%%%%%%
\chapter{Festen}

\section{Det var i vår ungdoms fagraste vår}\index{Det var i vår ungdoms fagraste vår}

...

\section{Portos visa}\index{Portos visa}
\textit{Mel: Annie get your gun}\\

Jag vill börja gasqua.\\
Var fan är min flaska\\
Vem i helvete stal min butelj?\\
Skall törsten mig tvinga\\
en TT börja svinga?\\
Men vad fan bara blunda och svälj!\\
Vilken smörja! Får jag spörja:\\
Vem för fan tror att jag är en älg?\\
Till England vi rider\\
och sedan vad det lider\\
träffar vi välan på någon pub.\\
Och där ska vi festa,\\
blott dricka av det bästa\\
utav whisky och portvin,\\
jag tänker gå hårt in\\
för att smaka på rubb och stubb.\\

Rubb och stubb, rubb och stubb, rubb å stubb,\\
rubb å stubb, rubb å...\\

\section{Jag var full en gång}\index{Jag var full en gång}
\textit{Mel: Flottarkärlek}\\

Jag var full en gång för länge sen\\
på knäna kröp jag hem\\
varje dike var för mig ett vilo hem\\
I varje hörn och varje vrå\\
hade jag en liten vän\\
ifrån renat upp till nittiosex procent.\\

Jag var full en gång för länge sen\\
på knäna kröp jag hem\\
och i sällskap hade jag en elefant.\\
Elefanten spruta vatten och jag trodde det var öl\\
sedan dess har alla kallar mig för knöl,\\
MERA ÖL!\\

\section{Härjarvisan}\index{Härjarvisan}
\textit{Mel: Gärdebylåten}\\

...

\section{Jag har aldrig vart på snusen}\index{Jag har aldrig vart på snusen}

...

%%%%%%%%%%%%%%%%%%%%%%%%%%%%%%%%%%%%%%%%%%%%%%%%%%%%%%%%%%%%%%%%%%%%%%%%%%%%%%%%
%%%%%%%%%%%%%%%%%%%%%%%%%%%%%%%%%%%%%%%%%%%%%%%%%%%%%%%%%%%%%%%%%%%%%%%%%%%%%%%%
%%%%%%%%%%%%%%%%%%%%%%%%%%%%%%%%%%%%%%%%%%%%%%%%%%%%%%%%%%%%%%%%%%%%%%%%%%%%%%%%
\chapter{Senare}

\section{See the little goblin}\index{See the little goblin}
\textit{(Sjungs under bordet.)}\\

See the little goblin,\\
see his little feet.\\
And his littel nosey-wose.\\
say is't the goblin sweet?\\
YES!

%%%%%%%%%%%%%%%%%%%%%%%%%%%%%%%%%%%%%%%%%%%%%%%%%%%%%%%%%%%%%%%%%%%%%%%%%%%%%%%%
%%%%%%%%%%%%%%%%%%%%%%%%%%%%%%%%%%%%%%%%%%%%%%%%%%%%%%%%%%%%%%%%%%%%%%%%%%%%%%%%
%%%%%%%%%%%%%%%%%%%%%%%%%%%%%%%%%%%%%%%%%%%%%%%%%%%%%%%%%%%%%%%%%%%%%%%%%%%%%%%%
\chapter{Öl}

\section{Strejk på Pripps}\index{Strejk på Pripps}
\textit{Mel: I natt jag drömde}\\

Inatt jag drömde något som,\\
jag aldrig drömt förut.\\
Jag drömde det var strejk på Pripps\\
och alla ölen var slut.\\
Jag drömde om en jättesal\\
där ölen stod på rad.\\
Jag drack så där ett 15 öl\\
och reste mig och sa:\\
Armen i vinkel\\
blicken i skyn\\
så var det menat\\
whisky och renat\\
vårt mål alkohol!\\
För dem som tål! -- SKÅL!!!\\

\section{Ju mera öl vi Dricker}\index{Ju mera öl vi Dricker}
\textit{Mel: Ju mer i är tillsammans}\\

Ju mera öl vi dricker,\\
vi dricker, vi dricker.\\
Ju mera öl vi dricker,\\
ju rundare vi bli.\\
För rundare är sundare\\
och sundare är rundare.\\
Ju mera öl vi dricker,\\
ju rundare vi bli.\\

%%%%%%%%%%%%%%%%%%%%%%%%%%%%%%%%%%%%%%%%%%%%%%%%%%%%%%%%%%%%%%%%%%%%%%%%%%%%%%%%
%%%%%%%%%%%%%%%%%%%%%%%%%%%%%%%%%%%%%%%%%%%%%%%%%%%%%%%%%%%%%%%%%%%%%%%%%%%%%%%%
%%%%%%%%%%%%%%%%%%%%%%%%%%%%%%%%%%%%%%%%%%%%%%%%%%%%%%%%%%%%%%%%%%%%%%%%%%%%%%%%
\chapter{Vin}

\section{Bordeaux, Bordeaux!}\index{Bordeaux, Bordeaux!}
\textit{Mel: I sommaren soliga dater}\\

Jag minns än idag hur min fader\\
Kom hem ifrån staden så glader\\
Och rada’ upp flaskor på rader\\
Och sade nöjd som så\\
``BORDEAUX, BORDEAUX!''\\

Han drack ett glas, kom i extas,\\
och sedan blev det stort kalas.\\
Och vi små glin, ja vi drack vin\\
som första klassens fyllesvin.\\
Och vi dansade runt där på golvet\\
och skrek så vi blev blå:\\
``Bordeaux, Bordeaux!''\\

\section{Feta fransyskor}\index{Feta fransyskor}
\textit{Mel: Marsche millitare (Tomrarnas julmarch)}\\

Feta fransyskor som svettas om fötterna,\\
de trampar druvor som sedan ska jäsas till vin.\\
Transpirationen viktig é,\\
ty den ge' fin bouquet.\\
Vårtor och svampar följer me’\\
men vad gör väl de’?\\

För... vi vill ha vin, vill ha vin, vill ha mera vin,\\
även om följderna blir att vi få lida pin.\\
Töser: Flaskan och glaset gått i sin.\\
Pågar: Hit med vin, mera vin!\\
Töser: Tror ni att vi är fyllesvin???\\
Pågar: JA! (fast större)\\

%%%%%%%%%%%%%%%%%%%%%%%%%%%%%%%%%%%%%%%%%%%%%%%%%%%%%%%%%%%%%%%%%%%%%%%%%%%%%%%%
%%%%%%%%%%%%%%%%%%%%%%%%%%%%%%%%%%%%%%%%%%%%%%%%%%%%%%%%%%%%%%%%%%%%%%%%%%%%%%%%
%%%%%%%%%%%%%%%%%%%%%%%%%%%%%%%%%%%%%%%%%%%%%%%%%%%%%%%%%%%%%%%%%%%%%%%%%%%%%%%%
\chapter{Öfrige dryckens viser}

\section{Spriteboa} \index{Spriteboa}
\textit{Mel: Snickerboa}\\

Till spritbolaget ränner jag\\
och bankar på dess port.\\
Jag vill ha nåt som bränner bra\\
och gör mig skitfull fort.\\
Expediten sade: ``Godda'\\
hur gammal kan min herre va'?\\
Har du nåt leg, ditt fula drägg,\\
kom hit igen när du fått skägg.''\\

Nej, detta var ju inte bra,\\
jag ska bli full ikväll.\\
Då plötsligt en idé jag fick\\
dom har ju sprit på Shell.\\
Många flaskor stod där på rad,\\
så nu kan jag bli full \& glad\\
Den röda drycken åker ner\\
nu kan jag inte titta mer!

\section{Ägnlaguck} \index{Ägnlaguck}
\textit{Mel: Änglahund}\\

\begin{enumerate}
\item Det var sent en kväll vi hadde dratt oss ner till Kåren.\\
Trött och slitna efter en blöt dag på Gubbholmen.\\
När vi hörde någon mummla nått om Gucken.\\
Där i gräset satt en man och en liten dunk.\\

\textit{Refr.}\\
Säg du genera, får vi smaka på den gucken.\\
Den är grön, och den har en touch av vin.\\
Man kan skymta, en smak av lite vodka.\\
Men det är härligt och vi svälger gärna mer.\\
Får man ta Gucken mer dig in på Kåren?\\
Den är god och den har en nörgrön färg.\\
Den är bärsk och satt, på endast 30 dar.\\
Får man det du general, då blir jag gla'.\\

\item Han svara, lite enkelt, som dom gör.\\
Din guck kommer säkert till Kåren, som sig bör.\\
Men du skall veta, att den måste snugglas in.\\
Jag slängde min plastmugg i skräpet och gick in.\\

\item Men ävem en guckdunk kan ta slut.\\
När det sista ut våra plastglas runnit ut.\\
Men har förströstan min vänner, den kommer åter.\\
ingen idé för er att ni sitter här och gråter.

\end{enumerate}


%%%%%%%%%%%%%%%%%%%%%%%%%%%%%%%%%%%%%%%%%%%%%%%%%%%%%%%%%%%%%%%%%%%%%%%%%%%%%%%%
%%%%%%%%%%%%%%%%%%%%%%%%%%%%%%%%%%%%%%%%%%%%%%%%%%%%%%%%%%%%%%%%%%%%%%%%%%%%%%%%
%%%%%%%%%%%%%%%%%%%%%%%%%%%%%%%%%%%%%%%%%%%%%%%%%%%%%%%%%%%%%%%%%%%%%%%%%%%%%%%%
\chapter{Glada sånger}

\section{Systéme internationale}\index{Systéme internationale}
\textit{Mel: Studentsången}\\

\begin{center}
W\ kg\ m\ Wb\ s\\
$\Omega$m\ T\ A\ rad\\
cd Sv N s\\
$\Omega$A M lx dB\\
\degree C W/m$^2$\\
J/kg H V C\\
kg/m$^3$ mol\\
m/s$^2$\\
m/s$^2$\\
F!
\end{center}

\section{Bruce's Philosophers Song}\index{Bruce's Philosophers Song}

Immanuel Kant was a real pissant\\
Who was very rarely stable\\

Heidegger, Heidegger was a boozy beggar\\
Who could think you under the table\\

David Hume could out-consume\\
Wilhelm Freidrich Hegel\\
(Live version: Schopenhauer and Hegel)\\

And Wittgenstein was a beery swine\\
Who was just as schloshed as Schlegel\\

There's nothing Nietzche couldn't teach ya\\
'Bout the raising of the wrist\\
Socrates, himself, was permanently pissed\\

John Stuart Mill, of his own free will\\
On half a pint of shandy was particularly ill\\

Plato, they say, could stick it away\\
Half a crate of whiskey every day\\

Aristotle, Aristotle was a bugger for the bottle\\
Hobbes was fond of his dram\\

And René Descartes was a drunken fart\\
I drink, therefore I am\\

Yes, Socrates, himself, is particularly missed\\
A lovely little thinker\\
But a bugger when he's pissed!\\

%%%%%%%%%%%%%%%%%%%%%%%%%%%%%%%%%%%%%%%%%%%%%%%%%%%%%%%%%%%%%%%%%%%%%%%%%%%%%%%%
%%%%%%%%%%%%%%%%%%%%%%%%%%%%%%%%%%%%%%%%%%%%%%%%%%%%%%%%%%%%%%%%%%%%%%%%%%%%%%%%
%%%%%%%%%%%%%%%%%%%%%%%%%%%%%%%%%%%%%%%%%%%%%%%%%%%%%%%%%%%%%%%%%%%%%%%%%%%%%%%%
\chapter{Ekivoka sånger}

\section{Balladen om Theobald-Thor}\index{Balladen om Theobald-Thor}
\textit{Mel: Ökänd}\\

\begin{enumerate}
\item En man som hette Theobald-Thor\\
han var en skicklig tamburmajor.\\
Succe'n han gjorde var alltid stor\\
för han snurra och slängde sin kuk.

\textit{Refr. (Eftger varje vers utom den sista.)}\\
För det var en stor kuk!\\
En lång kraftig och tung\\
från dess topp till dess rot\\
var den tre, fyra fot\\
och en medelstor ryggsäck till pung, pung, pung, pung, pungeli-pung, pung, pung.\\

\item En dag gick Theobald ut en stund\\
för att gå för sig själv i en lummig lund.\\
Då mötte han en söt liten dam med hund\\
som fick se honom svänga sin kuk.

\item Theobald prova ett trick han lärt\\
han släppte sin lem men en kraftig snärt.\\
I huvut' på hunden som avled tvärt\\
me'ns han snurra och svänga sin kuk.

\item Men damen hon blev helt bestört\\
och skrek och svort så herhört.\\
För det var ingen lyckat flirt\\
att snurra och svänga sin kuk.

\item Till följd av damens arga gnäll\\
han anhölls redan samma kväll\\
Och sattes i en ensamcell\\
att snurra och slänga sin kuk.

\item När molet kom i rätten opp\\
sa åklagarn: Det får bli ett stopp.\\
Man får ej vifta med sin snopp\\
och snurra och svänga sin kuk, fast

\item Men dommarn han var tollerant\\
Han sa: Själv gör jag likadant\\
Jag tycker det är intressant\\
att snurra och slänga sin kuk.\\
för att jag har också en stor...

\item Så Theobald släpptes fri\\
och liksom dommarn tycker vi\\
att tjejer dom ska skita i\\
om man snurrar och svänger sin kuk\\
För att vi har också en stor kuk!\\
En lång kraftig och tung\\
Från dess topp till dess rot\\
är den tre fyra fot\\
och en medelstor ryggsäck till pung, pung, pung, pungeli-pung, pung, pung.

\end{enumerate}

\cleardoublepage
\idxlayout{columns=1}
\printindex

\end{document}